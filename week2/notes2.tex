% Created 2023-02-12 Sun 14:58
% Intended LaTeX compiler: xelatex
\documentclass[11pt]{article}
\usepackage{graphicx}
\usepackage{longtable}
\usepackage{wrapfig}
\usepackage{rotating}
\usepackage[normalem]{ulem}
\usepackage{amsmath}
\usepackage{amssymb}
\usepackage{capt-of}
\usepackage{hyperref}
\usepackage{minted}
\usepackage{color}
\usepackage{listings}
\usepackage{tikz}
\usepackage[margin=1in]{geometry}
\usepackage{mathspec}
\setmainfont{Charis SIL}
\setmathsfont(Digits,Latin,Greek){Charis SIL}
\date{\today}
\title{}
\hypersetup{
 pdfauthor={},
 pdftitle={},
 pdfkeywords={},
 pdfsubject={},
 pdfcreator={Emacs 28.2.50 (Org mode 9.5.5)}, 
 pdflang={English}}
\begin{document}

\section*{Lecture 2}
\label{sec:orge716ee7}
\subsection*{Ownership model}
\label{sec:org524b179}
\subsubsection*{Double free and use after free}
\label{sec:org58c33fd}

Real issues in systems programming we will be solving.

C example:
\begin{minted}[]{c}
// make a string lowercase
char *to_lowercase(char *string);
\end{minted}

\vspace{2cm}
\begin{center}
  \begin{tikzpicture}[
      node/.style={align=center},
      end/.style={rectangle,draw=black,rounded corners},
      arrow/.style={text width=2cm},
    ]
    \node(A) at (0,0) {\Large What to free?};
    \node(B) at (-4,-3) {frees input?};
    \node(C) at (0,-3) {different?};
    \node[end](D) at (5,-3) {exactly one};
    \node[end](E) at (-6,-5) {return value};
    \node[end](F) at (-2,-5) {both};
    \draw[->] (A) -- node[arrow,left] {malloc new string} (B);
    \draw[->] (A) -- node[arrow,right,pos=0.6] {malloc only if different} (C);
    \draw[->] (A) -- node[arrow,right,yshift=5pt] {reuse string} (D);
    \draw[->] (C) -- node[above] {no} (D);
    \draw[->] (C) -- node[above] {yes} (B);
    \draw[->] (B) -- node[left] {yes} (E);
    \draw[->] (B) -- node[right] {no} (F);
  \end{tikzpicture}
\end{center}
\vspace{2cm}

\subsubsection*{Ownership rules}
\label{sec:orgc53494c}
\begin{enumerate}
\item Every value has an owner
variable/struct owns its data
\item That owner is unique
\item When the owner goes out of scope, the value is dropped
\end{enumerate}
\begin{minted}[]{rust}
{
    // string owns the string allocated on the heap
    let string: String = "Hello".to_string()
} // string freed automatically
\end{minted}

\subsubsection*{Moving}
\label{sec:org588c1b0}
transfer ownership, old value invalid

\begin{minted}[]{rust}
fn say_string(string: String) { // takes ownership of `string`
    println!("{}", string);
} // dropped here

let string: String = "Hello".to_string();
say_string(string); // ownership transferred from `string`
say_string(string); // can't use `string` anymore
\end{minted}

One solution:
\begin{minted}[]{rust}
fn say_string(string: String) {
    println!("{}", string);
}

let string = "Hello".to_string();
say_string(string.clone());
say_string(string);
\end{minted}
Ugh, this makes an extra allocation, but sometimes necessary


Rust equivalent to C code from before:
\begin{minted}[]{rust}
fn to_lowercase(string: String) -> String;
\end{minted}
What happens here?

\begin{itemize}
\item Double free
\label{sec:orgd3084c8}
No double free since owner unique

\item Use after free
\label{sec:org67060b6}
No use after free, since once ownership is transferred you can't access the value



\newpage
\end{itemize}
\subsubsection*{Borrowing}
\label{sec:org4beca48}
\begin{itemize}
\item References
\label{sec:org65e70be}
\(\approx\) pointers

\item Example
\label{sec:org980b986}
We don't want \texttt{say\_string} to take ownership of \texttt{string}
\begin{minted}[]{rust}
fn say_string(string: &String) {
    println!("{}", string);
}

let string = "Hello".to_string();
say_string(&string);
say_string(&string);
\end{minted}

\item Mutable borrowing
\label{sec:org48fc288}
\begin{minted}[]{rust}
struct Counter {
    count: u32,
}

impl Counter {
    fn get_count(&self) -> u32 {
        self.count
    }

    fn increment(&mut self) {
        self.count += 1;
    }
}
\end{minted}

\item Slices + XOR aliasing
\label{sec:org9d6d242}
\begin{minted}[]{rust}
let mut vector: Vec<i32> = vec![1, 2, 3, 4];
let two: &i32 = &vector[1];
vector.clear(); // uh oh!
\end{minted}
\end{itemize}
\end{document}